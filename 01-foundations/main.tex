 
\documentclass[10pt]{article}



\usepackage{amsmath}
\usepackage{amssymb}
\usepackage{amsfonts}
\usepackage{amsthm}
\usepackage{amscd}
\usepackage{mathtools}

\title{Probability and Stochastic Processes}
\author{Morgan Bergen}
\date{\today}

\begin{document}

\maketitle

\section[1]{Probability axioms}

A probability model assigns a number between 1 and 0 to every event.
The probability of the union of mutually exclusive evnts is the sum of the probabilities of the event in the union.
Set theory representation of a sample space S as the universal set, outcomes s that are the elements of the universal set S, and events A that are the sets of elements.  We must complete the model by incorporating P[A] to $\forall a\, \exists A$.

\subsection{Relative frequency notion of probability}
With respect to the physical idea of an experiment, the probability of an event is the proportion of time that event is observed in a large number of runs of the experiment.

\subsection{Axioms of probability}
Probability measure is a function that maps events in the sample space to $\mathbb{R}$   The entire theory of probability will be built upon the following three axioms. \\\\
$\textbf{Probability measure} \quad \mathcal{P}[\cdot] \mid 0\leq \mathcal{P} \in \mathbb{R} \leq 1$ \\

\subitem{\textbf{Axiom 1}}
$\quad \forall{A}, \mathcal{P}[A] \geq {0}$ \\

\subitem{\textbf{Axiom 2}}
$\quad \mathcal{P}[\mathcal{S}]=1$ \\

\subitem{\textbf{Axiom 3}} 
$\quad A_{1}, A_{2}, A_{3},\ldots,A_{n}$ \text{for any countable collection of mutually exclusive events} \\\\ 
$\exists \quad \mathcal{P}[A_{1} \cup A_{2} \cup A_3 \cup \ldots] =$ 
$\mathcal{P}[A_{1}] + \mathcal{P}[A_{2}] + \cdots + \mathcal{P}[A_{n}] $ \\

\subsection{Theorems of Probability}




\end{document}